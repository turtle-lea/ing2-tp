El valor de cada story fue calculado en función de su importancia dentro del sistema. 

Dentro de las stories con más valor se distinguen 2 grandes grupos:
\begin{itemize}
  \item Aquellas relacionadas con el simulador, ya que como fue mencionado anteriormente es prioritario para el cliente
  \item Aquellas indispensables para la aplicación, sin las cuales la misma deja de tener sentido. En este caso temas de registro y login de usuarios y creación y aceptación
  de desafíos
\end{itemize}

En el nivel de \emph{great} incluímos aquellas que le dan identidad y flexibilidad a nuestro juego. En este caso, relacionadas con las apuestas de fichas en los desafíos,
y las restricciones de capital a la hora de armar los equipos.

Las stories calificadas como \emph{good} son similares a las de \emph{great} pero de menor importancia.

Finalmente, el nivel de \emph{average} comprende tareas de back office, relacionadas con carga de datos. Y el \emph{nice to have} incluye detalles no tan relevantes que podrían mejorar la calidad.
