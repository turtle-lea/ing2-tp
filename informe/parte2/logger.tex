El log de simulación debe permitir registrar todos los eventos interesantes de la simulación como inicio y fin del partido o de un turno,
intento de movimiento y su resultado (exitoso o fallido), etc. Los eventos a loggear en principio serán siempre los mismos, pero eventualmente
podría cambiar la representación del log (un archivo, impresión por consola, un objeto especial, etc). Estas ideas están reflejadas en la
intención del patrón de diseño \emph{Builder}. En nuestro caso, el builder es el {\tt Logger} y el producto será una impresión por consola, un 
{\tt Archivo}, etc. El director está distribuído entre los movimientos defensivos y ofensivos, reboteo y turno.

A continuación presentamos el modelado de clases del log de simulación:

\begin{figure}[h!]
  \includegraphics[scale=0.35]{imagenes/clases-logger.png}
  \caption{Logger usando el patrón \emph{Builder}}
\end{figure}

El Turno es el director principal (al enviar el mensaje ``comenzar()'' este se encarga de ejecutar jugadas y movimientos). Adem\'as provee
el mensaje ``logger()'' para que los movimientos y jugadas puedan enviar mensajes al logger.

El logger se instancia desde el subsistema de desafíos. Cuando a un objeto desafío se le envía el mensaje ``simular()'', se realiza lo siguiente:


\begin{algorithmic} 
	\State ...
	\State Logger $logger$ = new LoggerEnArchivo()
	\State Simulador simulador = new SimuladorDeBasket()
	\State ResultadoDto res = simulador.simular(equipoDto1, equipoDto2, $logger$)
	\State Archivo logDeSimulacion = $logger$.archivo()
	\State ...
\end{algorithmic}

De esta manera obtiene el log final, que dependerá del logger que se haya utilizado.

