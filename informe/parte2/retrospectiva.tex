Los atrasos fueron causados principalmente por las actividades que los integrantes realizamos además de trabajar en este proyecto. Todos
nosotros tenemos trabajos de 6hs diarias. Adem\'as uno de los integrantes dedica 10 hs por semana a la docencia y el resto est\'a cursando otra materia
en la facultad. Haciendo un balance general y teniendo en cuenta horas de descanso, por semana cada uno tuvo disponible aproximadamente 10 horas (sin
contar los fines de semana). Por último hay que tener en cuenta que la coordinación grupal se complicó por la dificultad de coincidir todos en un mismo
horario.

En conclusión, las horas netas que cada uno tuvo disponibles fueron muy pocas (en las 3 semanas del sprint, cada uno tuvo aproximadamente unas 30 hs
disponibles más fines de semana). Además hubo una semana de disponibilidad cero por el parcial, por ende sólo tiene sentido considerar
dos semanas. Esto explica claramente los saltos en los burndowns.

Para llegar a tiempo a la entrega, en el avance del 20/04 se realizó una reunión grupal para pensar un diseño a nivel global y establecer parámetros
en común. Luego se repartieron stories iniciales para cada uno (dos por persona) y se avanzó en la parte central del simulador. Cada uno fue tomando
stories extra a medida que las terminaba o que se veía imposibilitado para avanzar.

Detectamos muchas dependencias entre stories. Esto provocó que en un punto las stories estaban completamente distribuidas pero ninguna había sido
completada porque se necesitaba integración con stories de otra persona. Es por esto que muchas stories se cerraron en los últimos días, mientras
que las horas fueron disminuyendo de manera más gradual.

Al unificar las partes, detectamos ciertos problemas a solucionar en un próximo sprint:
\begin{itemize}
 \item El logger no es cerrado para la modificación cuando se quieran agregar movimientos (por ejemplo: HacerFalta).
 \item Los parámetros deben ser modificados desde la clase que los usa. Esto deberíamos abstraerlo en algúna estructura especial que pueda
 ser modificada sin afectar a los que la usan.
 \item El jugador estrella (MVP) quedó fuera de esta iteración.
\end{itemize}
