En esta sección explicaremos cómo se combinan las capas de Desafíos y Simulador. Excluiremos de la explicación cómo resolver cuestiones de persistencia
y de interfaz de usuario ya que consideramos que serán dos capas externas que pueden ser modeladas de manera independiente usando interfaces para
comunicarse con la capa de desafíos.

Para comunicar capas usamos inversión de dependencias. De esta manera, ninguna de las capas depende de la otra y modificaciones hechas sobre una de ellas
no impactará en la otra.

El modelo de clases es el siguiente:

\newpage
\begin{landscape}

  \begin{figure}[h!]
   \includegraphics[scale=0.4]{imagenes/clases-capas.png}
   \caption{}
  \end{figure}

\end{landscape}
\newpage

Como se puede observar, para pasar información entre las capas usamos el concepto de DTO (Data Transfer Object). Este objeto en principio rompe la
biyección del modelo conceputal al mundo real, pero nos permite tener objetos que no pertenecen a ninguna de las capas y \emph{no usan} ninguna clase
dentro de las capas. De esta manera, ambas capas usan a estos objetos intermedios pero ahí se corta la cadena de dependencias. Tanto en la capa de desafíos
como en la de simulación habrá traductores que convertirán un DTO en el objeto apropiado según la capa, como por ejemplo:
\begin{itemize}
 \item En la capa de desafíos, el EquipoDto se traduce a un EquipoParticipante que se mapea al equipo que arma un participante en el juego.
 \item En la capa de simulación, el EquipoDto se traduce a un Equipo, que se mapea al equipo de una simulación de basket.
\end{itemize}

Además utilizamos el patrón \emph{Facade} para abstraer la funcionalidad del simulador en la clase {\tt SimuladorDeBasket}. Esta clase implementa
la interfaz {\tt Simulador} para hacer la inversión de dependencias y eliminar acoplamiento entre las capas.