Las jugadas ofensivas inicializan un colaborador interno portadorDelBalon al ser instanciadas, que representa el jugador que debe arrancar con la pelota.
Sin embargo, durante la ejecución de un turno pueden suceder eventos relacionados con la recuperacion del balon(intercepción exitosa u obtención de rebote)
que obliguen a los técnicos a elegir nuevamente una jugada para el mismo turno que sigue corriendo. Al analizar la jugada ofensiva
elegida no es posible asegurar que el portadorDelBalon inicial se corresponda con el portador del balon en el contexto de la ejecución del turno. 

Resulta más fácil explicar este hecho mirando el siguiente pseudocódigo, analizando por ejemplo los eventos al ocurrir un cambio de posesión:

~

\underline{En la clase \textbf{Turno}}

~

\begin{algorithmic}
	\Function{cambio\_de\_posesion}{equipo\_ofensivo, equipo\_defensivo}
	  \State $self.equipo\_ofensivo\gets equipo\_ofensivo$
	  \State $self.equipo\_defensivo\gets equipo\_defensivo$
	  \State $elegir\_jugadas$
	  \State $proxima\_accion$
	\EndFunction
\end{algorithmic}

~ 

Una acción que puede notificarle a un turno acerca de un cambio de posesion es por ejemplo una intercepción realizada por un jugador. Supongamos la intercepción del Alero de un equipo.

~

\underline{En la clase \textbf{Intercepcion}}

~

\begin{algorithmic}
	\Function{ejecutar}{unTurno}
	  \State \ldots
	  \State unTurno.cambio\_de\_posesion(self.jugador.equipo, self.jugador.equipo.oponente)
	  \State \ldots
	\EndFunction
\end{algorithmic}

~

En ese contexto, la acción inicial de la jugada elegida puede ser un pase entre el Base y el Escolta del equipo que ha interceptado recientemente la pelota (muchas jugadas
del basket empiezan con el base como portador inicial), cuando en realidad la intercepción pudo haber sido lograda por el Alero del equipo. Es como si la pelota se teletransportara
directamente al portadorDelBalon inicial de la jugada seleccionada.

Elegimos esta opción para facilitar la implementación y considerando el hecho de que lo mismo sucede para ambos equipos, por lo que no debería tener grandes repercusiones en la 
simulación de un desafío.

