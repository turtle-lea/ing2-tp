\subsection{Eligiendo jugada ofensiva}

Al momento de elegir una jugada ofensiva el técnico la selecciona de su libro de jugadas. El mismo tiene como colaboradores internos dos arreglos de 
GeneradoresDeJugada: uno para las ofensivas y otro para las defensivas. 

Introducimos una clase 'Generador*' por el siguiente motivo:
Las jugadas ofensivas tienen como colaborador interno a un jugador que indica quien es el portador actual del balón y opcionalmente otros colaboradores como la
cantidad de pases realizada hasta el momento. En el caso de que se produzca un reboteo en donde el mismo equipo 
atacante vuelva a ganar la posesión del balón la jugada debería resetearse. Es decir, volver a un estado inicial. 
En lugar de hacer eso, preferimos que cada vez que el técnico seleccione una jugada se genere una nueva instancia de la misma.
Para eso usamos precisamente un 'GeneradorDeJugadaX' donde X es una jugada tanto ofensiva como defensiva.

El generador tiene además como colaborador un objeto de clase 'frecuenciaDeUso', que es utilizado por un objeto generadorDeNumerosAleatorios que determinará 
de forma random la jugada seleccionada.

\begin{figure}[h!]
   \includegraphics[scale=0.45]{imagenes/elegir-jugada-clases.png}
   \caption{Diagrama de clases en donde se muestra el tecnico, su libro de jugadas, los generadores de las distintas jugadas y las clases de las jugadas}
\end{figure}

En resumen, los generadores son los encargados de generar las nuevas instancias de las jugadas elegidas por los técnicos. A continuación se muestra un diagrama
de secuencia en donde se muestran las interacciones que entran en juego desde el momento en que se le informa al técnico que debe elegir una jugada ofensiva.
El número aleatorio obtenido es menor a la frecuencia de uso del generadorDeJugadaOfensivaDe3PuntosKPases. El técnico utiliza dicha jugada el 70\% de las veces,
por lo que dicho generador es seleccionado
como jugada seleccionada por el técnico del equipo 'Los Pumas'. Finalmente se le envía el mensaje 'crearJugada' al generador para que devuelva una instancia
de JugadaOfensivaDe3PuntosKPases.

\newpage
\begin{landscape}

\begin{figure}[h!]
   \includegraphics[scale=0.27]{imagenes/elegir-ofensiva.png}
   \caption{Contexto: El técnico de Los Pumas elige una jugada ofensiva. El generador obtenido es el de JugadaOfensivaDe3PuntosKPases, el cual se utiliza para
   generar una JugadaOfensivaDe3PuntosKPases}
\end{figure}

\end{landscape}
\newpage

