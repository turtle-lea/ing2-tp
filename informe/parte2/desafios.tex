El subsistema de desaf\'ios modela los participantes, el armado de equipos con jugadores reales, apuestas con fichas y creaci\'on y ejecuci\'on
de desaf\'ios.

Los jugadores reales representan a los posibles jugadores que se pueden usar para armar equipos. Los Equipos reales son los equipos de la liga.

A continuaci\'on presentamos el diagrama de clases y explicaremos la resoluci\'on de los principales casos de uso.

\newpage
\begin{landscape}

  \begin{figure}[h!]
   \includegraphics[scale=0.35]{imagenes/clases-subsistema-desafios.png}
   \caption{}
  \end{figure}

\end{landscape}
\newpage


\subsubsection{Participante}
Un participante se crea con un {\tt Nombre}, {\tt Mail} y {\tt Password} que son esenciales al participante y serán inmutables en este modelo.
El método de creación de instancia es el siguiente:

\begin{algorithmic}
	\Function{Participante}{Nombre unNombre, Mail unMail, Password unaPassword}
	  \State nombre = unNombre
	  \State mail = unMail
	  \State password = unaPassword
	  \State cap = new Capital(CAP\_INICIAL)
	  \State fichasDisponibles = new CantidadDeFichas(FICHAS\_INICIALES)
	  \State equiposGuardados = [\ ]
	  \State desafiosAbiertos = [\ ]
	  \State desafiosGanados = [\ ]
	  \State desafiosPerdidos = [\ ]
	\EndFunction
\end{algorithmic}

CAP\_INICIAL y FICHAS\_INICIALES son constantes predefinidas en la clase.

\begin{figure}[h!]
   \includegraphics[width=\textwidth]{imagenes/clases-participante.png}
   \caption{Modelado de Participantes}
\end{figure}

\subsubsection{Armar equipos}
Para armar equipos existe la clase {\tt ConstructorDeEquipos} que contiene la lógica de control de capital. Se inicializa con un Participante (que
conoce su capital) y luego se
van eligiendo jugadores. Cada vez que se elige un jugador, se resta el costo del jugador al capital ($unJugadorReal.costo$) y se actualiza el saldo
restante en un contador interno. Si el jugador
es reemplazado por otro, primero se suma el costo del jugador que se saca y se resta el que se pone (usando los mensajes ``restar(Costo)'' y ``sumar(Costo)''
de la clase {\tt Capital}).

Si se intenta restar un costo mayor al capital disponible, el método de ``restar(Costo)'' lanza una excepción que será elevada al método que llamó
al {\tt ConstructorDeEquipos}, informando la imposibilidad de comprar ese jugador por falta de fondos.

El mensaje ``comprar()'' retorna el equipo armado.

Los jugadores reales, sus costos y estadísticas se pueden acceder a partir de los equipos reales {\tt EquipoReal} que representan los equipos de la 
liga real.

Una vez que el equipo fue armado, se puede usar para crear un desafío o para guardarlo para uso futuro. En el primer caso, se utiliza el mensaje
``unParticipante.crearDesafio(unEquipoParticipante, unaApuesta)'', mientras que en el último, se usa el
mensaje ``unParticipante.guardarEquipo(unEquipoParticipante)''.

\begin{figure}[h!]
   \includegraphics[width=\textwidth]{imagenes/clases-armar-equipos.png}
   \caption{Modelado del ConstructorDeEquipos}
\end{figure}

\begin{algorithmic}
	\Function{Participante::crearDesafio}{EquipoParticipante unEquipo, Apuesta unaApuesta}
	  \State desafiosAbiertos.add(new Desafio(unEquipo, unaApuesta))
	\EndFunction
\end{algorithmic}

\begin{algorithmic}
	\Function{Participante::guardarEquipo}{EquipoParticipante unEquipo}
	  \State equiposGuardados.add(unEquipo)
	\EndFunction
\end{algorithmic}

\subsubsection{Apuestas}
Las apuestas se modelan usando \emph{Null Object Pattern} para permitir desafíos sin apuesta. La {\tt ApuestaNula} tendrá un valor de 0 fichas.

\begin{figure}[h!]
   \includegraphics[width=0.5\textwidth]{imagenes/clases-apuestas.png}
   \caption{Modelado de Apuestas}
\end{figure}


\subsubsection{Crear y simular desafíos}
Al crear un desafío, como vimos anteriormente, se agrega a la lista de desafíos abiertos del participante creador. Ahora analizaremos el proceso de
creación de un desafío.

\begin{algorithmic}
	\Function{Desafio}{EquipoParticipante unEquipo, Apuesta unaApuesta}
	  \State equipoParticipanteDesafiante = unEquipo
	  \State apuesta = unaApuesta
	  \State estado = new EstadoDesafioAbierto(self)
	\EndFunction
\end{algorithmic}

Observar que el colaborador equipoParticipanteDesafiado quedó en null, pero esto no importa dado que ese equipo no se puede acceder desde
afuera (el {\tt EstadoDesafioAbierto} no lo permite).

Un desafío puede estar abierto, cerrado o terminado:
\begin{description}
 \item[Abierto:] Sólo responde los mensajes ``inscribirOponente'', ``apuesta'' y ``ganador?'' (que siempre devolverá False). El resto de los mensajes
 arrojarán excepciones pertinentes, dado que no tiene sentido que sean invocados en otro caso. Cuando se inscribe un oponente, el desafío deja de estar
 en la lista de desafios abiertos del participante y pasa a estar en estado cerrado, y se invoca a ``simular()''.
 \item[Cerrado:] Los mensajes que se responde son ``simular'' y ``equipoDesafiado'', el resto arrojarán execpciones. Dado que el flujo de ejecución no termina
 desde que se inscribe un oponente hasta el fin de la simulación, los desafíos en este estado duran muy poco tiempo (desde el punto de vista del oponente
 que se inscribe, cuando el mensaje retorna, la simulación ya finalizó). Luego se pasa al estado Finalizado, indicando previamente a ambos participantes
 si el desafío fue ganado o perdido (cada uno lo agregará a la lista que corresponda) y pagando apuestas de ser necesario.
 \item[Terminado:] Responde a todos los mensajes excepto ``inscribirOponente'' y ``simular''.
\end{description}

Esta idea de los estados se corresponde con el patrón de diseño \emph{State} ya que la clase Desafío cambia todo su comportamiento cuando
cambia de estado (como si cambiara de clase).

A continuación se muestra un escenario de inscripción y simulación de un desafío.

  \begin{figure}[h!]
   \includegraphics[width=\textwidth]{imagenes/desafios-inscribirOponente-obj.png}
   
   \caption{``Gastón (un participante) creó un desafío con su equipo llamado 'LosPumas' con una apuesta de 10 fichas y Martín se inscribe con su equipo llamado 
'AllBlacks'.'' La flecha roja es conocimiento débil pero luego de ejecutar ``cambiarEstado'' pasa a ser conocimiento fuerte.}
  \end{figure}
  
  \begin{figure}[h!]
   \includegraphics[width=\textwidth]{imagenes/desafios-simular-obj.png}
   \caption{``Siguiendo el escenario anterior, la simulación termina en que Gastón es ganador con Los Pumas con 50 puntos y Martín pierde con AllBlacks con 20 puntos.
Por ser la diferencia mayor a 20 puntos, Gastón gana 30 fichas (20 de la apuesta más 10 del premio base) y aumenta su cap en 1\%.''}
  \end{figure}

\newpage
\begin{landscape}

  \begin{figure}[h!]
   \includegraphics[scale=0.4]{imagenes/desafios-inscribirOponente.png}
   \caption{}
  \end{figure}

\end{landscape}
\newpage
\begin{landscape}

  \begin{figure}[h!]
   \includegraphics[scale=0.35]{imagenes/desafios-simular.png}
   \caption{}
  \end{figure}

\end{landscape}
\newpage

En el primer diagrama observar que cuando se llama a ``simular()'' del Desafío, el estado ya es EstadoDesafioCerrado (por eso en el segundo diagrama
el estadoDesafíoCerrado es un colaborador interno).

Cuando se termina de ejecutar la simulación, se informa a los participantes quién ganó y quién perdió, se paga la apuesta, en este caso se aumenta
el cap en un porcentaje y se cambia el estado del desafío a Terminado. Los valores parametrizables por el momento se modaleron como embebidos en la
clase EstadoDesafioCerrado (10 puntos de premio base, 1\% de aumento de cap, 20 puntos como resultado abultado).

Cabe aclarar que {\tt Participante.pagar(CantidadDeFichas)} asume que siempre se puede pagar (porque desde afuera pueden consultarse las fichas disponibles).
 En caso de no alcanzar las fichas, arroja una excepción.
