\subsection{Defendiendo un movimiento ofensivo}

Se puede pensar que en el basket existe una acción contraria a cada movimiento ofensivo. Para el pase existe la intercepción y para el tiro el bloqueo. (El rebote no es considerado
movimiento ofensivo). Por lo tanto, las jugadas defensivas tienen la oportunidad de responder con una secuencia de acciones defensivas (puede ser una o muchas) ante cada movimiento
ofensivo. En el caso de una jugada defensiva hombre a hombre, cada acción de un jugador ofensivo en una posición determinada es defendida por una única acción 
contraria realizada por el jugador en la misma posición del equipo rival. Sin embargo, podría existir una jugada defensiva que marque únicamente los tiros de 3 puntos pero cada uno de ellos
con varios jugadores. Es por eso que cada acción ofensiva puede ser defendida con una serie de movimientos.

La jugada defensiva recibe el mensaje defender, pero en un principio desconoce \emph{qué} acción debe defender. Para evitar el uso de \textbf{if} se utiliza el double dispatch.
Al objeto pasado por parámetro se le envía el mensaje \emph{informarTipoDeMovimiento}.

\newpage
\begin{landscape}

  \begin{figure}[h!]
   \includegraphics[scale=0.35]{imagenes/defender-pase.png}
   \caption{}
  \end{figure}

\end{landscape}
\newpage

