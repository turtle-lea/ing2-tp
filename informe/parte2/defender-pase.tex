\subsection{Defendiendo un movimiento ofensivo}

Se puede pensar que en el basket existe una acción contraria a cada movimiento ofensivo. Para el pase existe la intercepción y para el tiro el bloqueo. (El rebote no es considerado
movimiento ofensivo). Por lo tanto, las jugadas defensivas tienen la oportunidad de responder con una secuencia de acciones defensivas (puede ser una o muchas) ante cada movimiento
ofensivo. En el caso de una jugada defensiva hombre a hombre, cada acción de un jugador ofensivo en una posición determinada es defendida por una única acción 
contraria realizada por el jugador en la misma posición del equipo rival. Sin embargo, podría existir una jugada defensiva que marque únicamente los tiros de 3 puntos pero cada uno de ellos
con varios jugadores. Es por eso que cada acción ofensiva puede ser defendida con una serie de movimientos.

La jugada defensiva recibe el mensaje defender, pero en un principio desconoce \emph{qué} acción debe defender. Para evitar el uso de \textbf{if instanceOf} 
sobre el tipo del movimiento ofensivo: pase, tiro de 2 puntos, tiro de 3 puntos se utiliza \emph{double dispatch}.
Al objeto pasado por parámetro se le envía el mensaje \emph{informarTipoDeMovimiento}. 

Las jugadas defensivas utilizan fuertemente las posiciones de los jugadores atacantes. Por lo tanto, definimos una clase abstracta \textbf{JugadaDefensiva} que incluye los mensajes
que toda jugada defensiva debería saber responder. Saber responder estos mensajes no significa que deba defenderse con alguna acción. La secuencia de acciones a devolver puede ser 
vacía ([ ]). Por ejemplo, en el caso de marcar únicamente los tiros de 3 puntos, los mensajes 'defenderTiroPor2PuntosDe*' 
deberían responderse con un array vacío ([ ]).

\begin{figure}[h!]
  \includegraphics[scale=0.25]{imagenes/diagrama-clases-defensiva.png}
  \caption{Diagrama de clases relacionado con las jugadas defensivas y sus colaboraciones}
\end{figure}

En el caso de JugadaDefensivaHombreAHombre, nuevamente se utiliza \emph{double dispatch}, en este caso para evitar el uso de un \textbf{if} del estilo \textbf{instanceOf PosicionBase}.
De esta manera se obtiene la posición del jugador que ha originado la acción y responde con una única acción defendida por el jugador del equipo rival.
El siguiente diagrama de secuencia representa la forma en que se obtiene la respuesta defensiva del equipo rival a \emph{Los Pumas} correspondiente a un pase entre 
el base Martín y el escolta Gastón. La respuesta obtenida es un arreglo con un único elemento que contiene una intercepción realizada por el base del equipo rival.

\newpage
\begin{landscape}

\begin{figure}[h!]
   \includegraphics[scale=0.35]{imagenes/defender-pase.png}
   \caption{Contexto: La jugada defensiva del equipo rival calcula la serie de movimientos defensivos en respuesta a un pase hecho por Martín, el base de Los Pumas}
\end{figure}

\end{landscape}
\newpage


