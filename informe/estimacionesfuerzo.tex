Una vez determinadas las user stories, empezamos a estimar el esfuerzo en story points. Para esto utilizamos la técnica de Planning Poker.

Primero realizamos la estimación de las historias asociadas al simulador. Cada integrante del equipo eligió un conjunto de historias 
que consideró de 1 punto de esfuerzo. Luego realizamos una puesta en común en la que estuvimos de acuerdo por unanimidad. A continuación
analizamos todas las tarjetas restantes relacionadas con la simulación y acordamos un puntaje utilizando Planning Poker. La técnica
fue muy útil para llegar a un acuerdo, ya que permitió evaluar posturas muy distintas sobre las historias. En algunos casos hubo acuerdos casi
unánimes y en otros dedicamos bastante tiempo a la discusi\'on.

Luego estimamos las historias por fuera del simulador. También comenzamos eligiendo una con esfuerzo de 1 punto y a partir de esa estimamos
el resto hasta completar el proceso.
