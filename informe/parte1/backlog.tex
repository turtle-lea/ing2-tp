A continuación presentamos las user stories que forman parte de el backlog del proyecto, pero que no entran en el primer sprint.

Para mayor claridad, fueron ordenadas según categorías funcionales y por valor de negocio (de mayor a menor).

\subsection{Registro y autenticación de usuarios}

\simplestory
{337: Manejo de usuarios}
{Como diseñador del juego quiero que los usuarios puedan registrarse, loguearse y desloguearse de la aplicación para incrementar la seguridad la aplicación}
{5 pt.}
{Must Have}
{-}

\subsection{Desaf\'ios}

\simplestory
{231: Crear desafío}
{Como participante quiero crear un desafío para aparecer en el listado de desafíos disponibles.}
{3 pt.}
{Must Have}
{-}

\vspace{1cm}

\simplestory
{239: Aceptar desafío}
{Como participante desafiado quiero aceptar un desafío para ganar fichas e ir aumentando el ranking dentro del sistema.}
{3 pt.}
{Must Have}
{-}

\vspace{1cm}

\simplestory
{261: Ver resultados por desfío}
{Como participante quiero ver la cantidad de puntos que realizó cada equipo en la simulación de un desafío (score) y que se indique el ganador para saber qué tan bien o mal funcionó el equipo y observar a simple vista si gané o perdí.
}
{2 pt.}
{Great}
{-}

\vspace{1cm}

\simplestory
{241}
{Como participante quiero ver tabla de posiciones en base a desafíos ganados/perdidos para compararme con otros participantes.}
{2 pt.}
{Good}
{-}

\vspace{1cm}

\subsection{Equipos}

\simplestory
{230: Armar equipo}
{Como participante quiero armar un equipo de basket con un tecnico y un nombre para utilizarlo en desafíos.}
{5 pt.}
{Must Have}
% No queda muy lindo el itemize adentro de la tabla...
{\begin{itemize}
\item El equipo debe tener 5 jugadores (base, escolta, alero, ala pivot, pivot) elegidos de una lista con todos los jugadores de las ligas.
\item Cada jugador de la lista permitirá acceder a su costo (valor ficticio) y a sus estadísticas.
\item Se deberá validar que no se supere el capital informado por la aplicación.
\item Se deberá poder ver las jugadas que tiene asociadas cada técnico
\end{itemize}}

\vspace{1cm}

\simplestory
{267: Guardar equipo}
{Como participante quiero guardar equipos creados en mi lista de equipos y reutilizarlos para no tener que volver a armar 2 veces el mismo equipo}
{1 pt.}
{Good}
{-}

\vspace{1cm}

\simplestory
{237: Jugador estrella}
{Como participante quiero determinar el jugador estrella de mi equipo para personalizar mi equipo y maximizar la ventaja en la simulación del partido}
{1 pt.}
{Good}
{-}

\vspace{1cm}

\subsection{Capital}

\simplestory
{234: Elección de jugadores}
{Como diseñador del juego quiero realizar una carga del valor de los jugadores y sus datos estadísticos para que los participantes armen sus equipos}
{2 pt.}
{Great}
{\begin{itemize}
\item El valor de los jugadores no se volverá a modificar
\item Los porcentajes FG y 3P deben estar entre 0 y 100, APG $>=$ 0, RPG $>=$ 0, BPG $>=$ 0, SPG $>=$ 0, TO $>=$ 0, PPG $>=$ 0.
\item Inicialmente utilizar 100 como cota superior para todas las estadísticas. Sujeto a cambios
\end{itemize}}

\vspace{1cm}

\vspace{1cm}

\simplestory
{243: Capital disponible}
{Como participante quiero poder ver/conocer mi cap para determinar los límites monetarios del equipo que puedo armar.}
{1 pt.}
{Great}
{-}

\vspace{1cm}

\simplestory
{246: Aumento de cap por victoria abultada}
{Como diseñador del juego quiero que al ganar un desafio por una diferencia abultada, el dueño del equipo reciba un pequeño aumento de su cap para para recompensar las elecciones tomadas por el apostador.}
{1 pt.}
{Great}
{\begin{itemize}
\item Diferencia abultada = más de 20 puntos
\item Aumento de cap inicialmente es del 1\%, pero puede estar sometido a variaciones
\end{itemize}}

\vspace{1cm}

\subsection{Fichas y Apuestas}

\simplestory
{245: Premio en fichas por desafío ganado}
{Como diseñador del juego quiero que al ganar un desafío, el dueño del equipo vencedor se lleve un premio base de fichas (independientemente de si se apostó o no) y el total de fichas apostadas para recompensar al usuario ganador.
}
{2 pt.}
{Great}
{\begin{itemize}
\item Podría cambiar la manera de calcular ese ``premio base'' (por ejemplo podría dejar de ser fijo y ser una fórmula en función de algo), debería ser transparente para el sistema.
\item Se le asigna a los participantes una cantidad inicial de fichas
\item Los participantes pueden consultar sobre la cantidad de fichas disponible
\end{itemize}}

\vspace{1cm}

\simplestory
{241: Ranking cantidad de desafíos ganados}
{Como participante quiero ver tabla de posiciones en base a cantidad de fichas ganadas en apuestas para compararse con otros participantes.}
{2 pt.}
{Good}
{-}

\vspace{1cm}

\subsection{Datos estadísticos de jugadores}

\simplestory
{263: Carga de jugadores automática}
{Como diseñador del simulador quiero cargar jugadores y sus datos estadísticos automáticamente desde una página o servicio web para que haya jugadores en el sistema.}
{8 pt.}
{Nice to Have}
{\begin{itemize}
\item Validaciones de los datos ingresados
\item Parsear los datos de la fuente correspondiente
\item Detalles de comunicación
\end{itemize}}

\vspace{1cm}

\simplestory
{268: Datos de twitter}
{Como diseñador del simulador quiero que se obtengan datos de popularidad en Twitter de los jugadores para afectar las fórmulas de umbral de éxito.}
{13 pt.}
{Nice to Have}
{-}

\vspace{1cm}

\subsection{Libro de Jugadas}
\simplestory
{265}
{Como diseñador del simulador quiero crear técnicos asociándolos con un libro de jugadas para que haya técnicos en el sistema}
{2 pt.}
{Average}
{-}

\vspace{1cm}
