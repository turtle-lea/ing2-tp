\documentclass[a4paper]{article}
\usepackage[spanish]{babel}
\usepackage[utf8]{inputenc}
\usepackage{graphicx}
\usepackage{pdfpages}
\usepackage{enumerate}
\usepackage{listings}
\usepackage{color}
\usepackage{indentfirst}
\usepackage{fancyhdr}
\usepackage{latexsym}
\usepackage[colorlinks=true, linkcolor=black]{hyperref}
\usepackage{wrapfig}
\usepackage{calc}
\usepackage{amsmath, amsthm, amssymb}
\usepackage{amsfonts}
\definecolor{gray}{gray}{0.5}
\definecolor{light-gray}{gray}{0.95}
\definecolor{orange}{rgb}{1,0.5,0}

\usepackage{fancyhdr}
\pagestyle{fancy}

%\renewcommand{\chaptermark}[1]{\markboth{#1}{}}
\renewcommand{\sectionmark}[1]{\markright{\thesection\ - #1}}

\fancyhf{}

\fancyhead[LO]{Sección \rightmark} % \thesection\
\fancyfoot[LO]{\small{Leandro Matayoshi, Matías Pizzagalli, Gastón Requeni, Martín Santos}}
\fancyfoot[RO]{\thepage}
\renewcommand{\headrulewidth}{0.5pt}
\renewcommand{\footrulewidth}{0.5pt}
\setlength{\hoffset}{-0.8in}
\setlength{\textwidth}{16cm}
%\setlength{\hoffset}{-1.1cm}
%\setlength{\textwidth}{16cm}
\setlength{\headsep}{0.5cm}
\setlength{\textheight}{25cm}
\setlength{\voffset}{-0.7in}
\setlength{\headwidth}{\textwidth}
\setlength{\headheight}{13.1pt}

\renewcommand{\baselinestretch}{1.1}  % line spacing


\usepackage{underscore}
\usepackage{caratula}
\usepackage{url}

\newcommand{\cod}[1]{{\tt #1}}
\newcommand{\negro}[1]{{\bf #1}}
\newcommand{\ital}[1]{{\em #1}}
\newcommand{\may}[1]{{\sc #1}}
\newcommand{\tab}{\hspace*{2em}}

\newcommand{\sprintstory}[6]{\begin{tabular}{| p{3cm} | p{12cm} |}
 \hline
 Número: & #1 \\
 \hline
 User Story: & #2 \\
 \hline
 Esfuerzo estimado: & #3 \\
 \hline
 Business Value: & #4 \\
 \hline
 Descripción: & #5 \\
 \hline
 Criterios de\newline Aceptación: & #6 \\
 \hline
\end{tabular}}

\newcommand{\simplestory}[4]{\begin{tabular}{| p{3cm} | p{12cm} |}
 \hline
 Número: & #1 \\
 \hline
 User Story: & #2 \\
 \hline
 Esfuerzo estimado: & #3 \\
 \hline
 Business Value: & #4 \\
 \hline
\end{tabular}}

\newenvironment{taskstable}
{ \begin{tabular}{| p{14cm} | p{1cm} |}
 \hline
 \multicolumn{2}{|c|}{{\bf División en tareas}}\\
 \hline
 {\bf Tarea} & {\bf HH}\\
 \hline }
{ \end{tabular} }

\newcommand{\task}[2]{#1 & #2\\
 \hline}

\hypersetup{
 pdfstartview= {FitH \hypercalcbp{\paperheight-\topmargin-1in-\headheight}},
 pdfauthor={Grupo},
 pdfsubject={Dise\~{n}o}
}

\lstset{escapechar=@}

\begin{document}

\thispagestyle{empty}
\materia{Ingeniería de Software II}
\submateria{Primer Cuatrimestre de 2016}
\titulo{Trabajo Práctico I: The Curry Game}

\integrante{Leandro Matayoshi}{79/11}{leandro.matayoshi@gmail.com}
\integrante{Matías Pizzagalli}{257/12}{matipizza@gmail.com}
\integrante{Gastón Requeni}{400/11}{grequeni@hotmail.com}
\integrante{Martín Santos}{413/11}{javiersm00@gmail.com}

\makeatletter

\maketitle

\newenvironment{myindentpar}[1]
{\begin{list}{1}
         {\setlength{\leftmargin}{#1}}
         \item[]
}
{\end{list}}

\newpage


%%%%% EJEMPLOS DE USER STORY Y TAREAS

\sprintstory
{1}
{Como usuario no registrado quiero ingresar mi nombre, mi cuenta de email y una contraseña para poder registrarme en la aplicación.}
{3 pt.}
{Nice to Have}
{-}
{\begin{itemize}
  \item Ingresar mi nombre, mi cuenta de email y una contraseña, y apretar un botón que confirme el registro.
  \item No debe permitir ingresar un mail sintácticamente inválido.
  \item No debe permitir registrar dos veces el mismo mail.
  \item La contraseña debe ser ingresada dos veces y el sistema validar que sean idénticas.
\end{itemize}}

\vspace{1cm}

\simplestory
{1}
{Como usuario no registrado quiero ingresar mi nombre, mi cuenta de email y una contraseña para poder registrarme en la aplicación.}
{3 pt.}
{Nice to Have}

            
\begin{taskstable}
 \task
 {Como usuario no registrado quiero ingresar mi nombre, mi cuenta de email y una contraseña para poder registrarme en la aplicación}
 {3}
 
 \task
 {Como usuario no registrado quiero ingresar mi nombre, mi cuenta de email y una contraseña para poder registrarme en la aplicación}
 {3}
\end{taskstable}

%%%%%%%%%%%%%%%%%%%%%%%%%%%%%%%%%%%%%%%%%%%%%%%%%%%%%%%%%%%%%%%%%%%%%%%%%%%%%%%%%%%%%%%%%%%%%







\newpage

\bibliographystyle{plain}
\bibliography{tp3}

\end{document}
